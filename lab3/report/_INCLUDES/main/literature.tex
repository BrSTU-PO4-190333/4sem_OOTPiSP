\section*{Список использованных источников} % Секция без номера
\addcontentsline{toc}{section}{Список использованных источников} % Добавить в содержание

\begin{enumerate}
    \item [1.] Наследование в ООП пример. Что такое наследование. Для чего нужно наследование классов. ООП. C++ \#98
    \\\url{https://www.youtube.com/watch?v=O7ruEWCa7zc}
    \item [2.] Модификаторы доступа при наследовании. private public protected Спецификаторы доступа. ООП. C++ \#99
    \\\url{https://www.youtube.com/watch?v=6udKffus77A}
    \item [3.] Перегрузка операторов пример. ООП. Перегрузка оператора присваивания. C++ Для начинающих. Урок\#83
    \\\url{https://www.youtube.com/watch?v=nMM98LVJn-U}
    \item [4.] Перегрузка оператора равенства == и не равно !=. Перегрузка логических операторов сравнения. C++ \#84
    \\\url{https://www.youtube.com/watch?v=UsezbK-3BL0}
    \item [5.] Дружественные функции и классы пример. Для чего используются. Как определяются. Для двух классов \#88
    \\\url{https://www.youtube.com/watch?v=Ic19I0kcBnU}
    \item [6.] Дружественный метод класса. ООП. friend c++ что это. Функции друзья. C++ Для начинающих. Урок \#90
    \\\url{https://www.youtube.com/watch?v=c3FJv4v7NlU}
    \item [7.] Знания полученые от лабораторной работы №4 "Стэки и очереди" по дисциплине Алгоритмы и Структуры Данных (БрГТУ ПОИТ)
    \item [8.] Перегрузка оператора индексирования. Перегрузка операторов пример. C++ Для начинающих. Урок \#87
    \\\url{https://www.youtube.com/watch?v=f-N4QsyLluM}
    \item [9.] Виртуальные методы класса c++. Ключевое слово virtual. Ключевое слово override. ООП. C++ \#103
    \\\url{https://www.youtube.com/watch?v=YlbFPAugFNA}
\end{enumerate}