\paragraph{Основное содержание работы}

Написать программу, в которой создается иерархия классов. Включить полиморфные объекты в связанный список, используя статические компоненты класса. Показать использование виртуальных функций.

\paragraph{Порядок выполнения работы}

\begin{enumerate}
    \item[1.] Определить иерархию классов (в соответствии с вариантом).
    \item[2.] Определить в классе статическую компоненту - указатель на начало связанного списка объектов и статическую функцию для просмотра списка.
    \item[3.] Реализовать классы.
    \item[4.] Написать демонстрационную программу, в которой создаются
    объекты различных классов и помещаются в список, после чего список
    просматривается.
    \item[5.] Сделать соответствующие методы не виртуальными и посмотреть, что будет.
    \item[6.] Реализовать вариант, когда объект добавляется в список при создании, т.е. в конструкторе (смотри пункт 6 следующего раздела).
\end{enumerate}

\paragraph{Содержание отчета}

\begin{enumerate}
    \item[1.] Титульный лист: название дисциплины; номер и наименование работы; фамилия, имя, отчество студента; дата выполнения.
    \item[2.] Постановка задачи. Следует дать конкретную постановку, т.е. указать, какие классы должны быть реализованы, какие должны быть в них конструкторы, компоненты-функции и т.д.
    \item[3.] Иерархия классов в виде графа.
    \item[4.] Определение пользовательских классов с комментариями.
    \item[5.] Реализация конструкторов с параметрами и деструктора.
    \item[6.] Реализация методов для добавления объектов в список.
    \item[7.] Реализация методов для просмотра списка.
    \item[8.] Листинг демонстрационной программы.
    \item[9.] Объяснение необходимости виртуальных функций. Следует показать, какие результаты будут в случае виртуальных и не виртуальных функций.
\end{enumerate}

\paragraph{Приложение. Варианты заданий.}

Перечень классов:

\begin{enumerate}
    \item[1.] студент, преподаватель, персона, завкафедрой;
    \item[2.] служащий, персона, рабочий, инженер;
    \item[3.] рабочий, кадры, инженер, администрация;
    \item[4.] деталь, механизм, изделие, узел;
    \item[5.] организация, страховая компания, судостроительная компания, завод;
    \item[6.] журнал, книга, печатное издание, учебник;
    \item[7.] тест, экзамен, выпускной экзамен, испытание;
    \item[8.] место, область, город, мегаполис;
    \item[9.] игрушка, продукт, товар, молочный продукт;
    \item[10.] квитанция, накладная, документ, чек;
    \item[11.] автомобиль, поезд, транспортное средство, экспресс;
    \item[12.] двигатель, двигатель внутреннего сгорания, дизель, турбореактивный двигатель;
    \item[13.] республика, монархия, королевство, государство;
    \item[14.] млекопитающие, парнокопытные, птицы, животное;
    \item[15.] корабль, пароход, парусник, корвет.
\end{enumerate}