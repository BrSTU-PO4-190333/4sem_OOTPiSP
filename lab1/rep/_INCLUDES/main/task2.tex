\paragraph{6.} Программа решения задания 2

\lstinputlisting[
    name=main.cpp,
    language=C++,
]
{../../src/task2/option5/src/main.cpp}

\newpage

\paragraph{7.} Результат работы программы

\begin{lstlisting}[
    language=Out,
]
1) - -n-++m
	n = 1
	m = 1
- -n-++m = 
	= (- - n) - (++m) = 
	= n - (m+1) = 
	= 1 - (1+1) = 1 - 2 = -1
	= 1 - (1+1) = 1 - 2 = -1
Answer: -1

2) m*n<n++
	n = 1
	m = 1
m*n<n++
	(m * n) < n
	(1 * 1) < 1
	1 < 1
Answer: 0

3) n-- > m++
	n = 1
	m = 1
n-- > m++
	n > m
	1 > 1
Answer: 0
\end{lstlisting}

\paragraph{8.} Объяснение результатов

Запись минус-пробел-минус - это не инфиксная запись отнимания один, это двойное отрицание.

\begin{lstlisting}[
    language=empty,
]
x = 1
- - x //1
--x //0
\end{lstlisting}

Инфиксная форма: ++x - увеличение значения x на 1, и возврат увеличенный на 1.

\begin{lstlisting}[
    language=empty,
]
x = 1
a = ++x
print(a) //2
print(x) //2
\end{lstlisting}

Постфиксная форма: x++ - увеличивает значение x на 1, но возвращает значение обычного не увеличенного x.

\begin{lstlisting}[
    language=empty,
]
x = 1
a = x++
print(a) //1
print(x) //2
\end{lstlisting}