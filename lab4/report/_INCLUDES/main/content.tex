\begin{center}
    \textbf{\titlePageWorkType~№\titlePageWorkNumber}
\end{center}

\textbf{Тема}: <<\titlePageTopic>>

\textbf{Цель работы}: Получить практические навыки создания абстрактных типов данных и перегрузки операций в языке С++.

\begin{center}
    \textbf{Ход работы}:
\end{center}

\textbf{Порядок выполнения работы.}

\begin{enumerate}
    \item [1.] Выбрать класс абстрактного типа данных (АТД) в соответствии с вариантом.
    \item [2.] Определить и реализовать в классе конструкторы, деструктор, функции Input (ввод с клавиатуры) и Print (вывод на экран), перегрузить операцию присваивания. 
    \item [3.] Написать программу тестирования класса и выполнить тестирование. 
    \item [4.] Дополнить определение класса заданными перегруженными операциями ( в соответствии с вариантом).
    \item [5.] Реализовать эти операции. Выполнить тестирование.
\end{enumerate}

\begin{center}
    \textbf{Вариант 5}:
\end{center}

АТД - множество с элементами типа char.
Дополнительно перегрузить следующие операции:
\begin{itemize}
    \item () - конструктор множества (в стиле конструктора Паскаля);
    \item + - объединение множеств;
    \item <= - сравнение множеств.
\end{itemize}

\lstinputlisting[language=C++, name=main.cpp,]
{../src/option5/src/main.cpp}

\begin{lstlisting}[language=Out,]
0x7fffa24d6780 constructor
x1
[ a, b, c, ]

0x7fffa24d6770 constructor
x2
[ a, c, d, ]

(x1 <= x2) = 1

(x2 <= x1) = 0

(x1 <= x1) = 1

x3
0x7fffa24d6760 constructor
[ d, g, u, w, ]

x4
0x7fffa24d6750 constructor
[ d, l, s, t, ]

x5 = x3 + x4
0x7fffa24d6740 constructor
[ d, g, l, s, t, u, w, ]

0x7fffa24d6740 destructor
0x7fffa24d6750 destructor
0x7fffa24d6760 destructor
0x7fffa24d6770 destructor
0x7fffa24d6780 destructor
\end{lstlisting}