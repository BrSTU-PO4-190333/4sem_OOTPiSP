\paragraph{} \textbf{Порядок выполнения работы}

Написать и отладить три программы.
\textbf{Первая программа} демонстрирует использование контейнерных классов для хранения встроенных типов данных.
\textbf{Вторая программа} демонстрирует использование контейнерных классов для хранения пользовательских типов данных.
\textbf{Третья программа} демонстрирует использование алгоритмов STL.

\textbf{В программе №1} выполнить следующее:

\begin{enumerate}
    \item [1.] Создать объект-контейнер в соответствии с вариантом задания и заполнить его данными, тип которых определяется вариантом задания.
    \item [2.] Просмотреть контейнер.
    \item [3.] Изменить контейнер, удалив из него одни элементы и заменив другие.
    \item [4.] Просмотреть контейнер, используя для доступа к его элементам итераторы.
    \item [5.] Создать второй контейнер этого же класса и заполнить его данными того же типа, что и первый контейнер.
    \item [6.] Изменить первый контейнер, удалив из него n элементов после заданного и добавив затем в него все элементы из второго контейнера.
    \item [7.] Просмотреть первый  и второй контейнеры.
\end{enumerate}

\textbf{В программе №2} выполнить то же самое, но для данных пользовательского типа.

\textbf{В программе №3} выполнить следующее:

\begin{enumerate}
    \item [1.] Создать контейнер, содержащий объекты пользовательского типа. Тип контейнера выбирается в соответствии с вариантом задания.
    \item [2.] Отсортировать его по убыванию элементов.
    \item [3.] Просмотреть контейнер.
    \item [4.] Используя подходящий алгоритм, найти в контейнере элемент, удовлетворяющий заданному условию.
    \item [5.] Переместить элементы, удовлетворяющие заданному условию в другой (предварительно пустой) контейнер. Тип второго контейнера определяется вариантом задания.
    \item [6.] Просмотреть второй контейнер.
    \item [7.] Отсортировать первый и второй контейнеры по возрастанию элементов.
    \item [8.] Просмотреть их.
    \item [9.] Получить третий контейнер путем слияния первых двух.
    \item [10.] Просмотреть третий контейнер.
    \item [11.] Подсчитать, сколько элементов, удовлетворяющих заданному условию, содержит третий контейнер.
    \item [12.] Определить, есть ли в третьем контейнере элемент, удовлетворяющий заданному условию.
\end{enumerate}

\begin{table}[!htp]
    \centering
    \caption{Приложение. Варианты заданий.}
    \label{table:options}

    \begin{tabular}{|c|c|c|c|} 
        \hline
        № п/п & Первый контейнер & Второй контейнер & Встроенный тип данных \\ \hline
        \hline
        1     & vector           & list             & int                   \\ \hline
        2     & list             & deque            & long                  \\ \hline
        3     & deque            & stack            & float                 \\ \hline
        4     & stack            & queue            & double                \\ \hline
        5     & queue            & vector           & char                  \\ \hline
        6     & vector           & stack            & string                \\ \hline
        7     & map              & list             & long                  \\ \hline
        8     & multimap         & deque            & float                 \\ \hline
        9     & set              & stack            & int                   \\ \hline
        10    & multiset         & queue            & char                  \\ \hline
        11    & vector           & map              & double                \\ \hline
        12    & list             & set              & int                   \\ \hline
        13    & deque            & multiset         & long                  \\ \hline
        14    & stack            & vector           & float                 \\ \hline
        15    & queue            & map              & int                   \\ \hline
        16    & priority\_queue   & stack            & char                 \\ \hline
        17    & map              & queue            & char                  \\ \hline
        18    & multimap         & list             & int                   \\ \hline
        19    & set              & map              & char                  \\ \hline
        20    & multiset         & vector           & int                   \\ \hline
    \end{tabular}
\end{table}